% LaTeX file for a 1 page document
\documentclass[12pt]{article}

\title{Complement set for varieties of ideals with single generator}
\author{Xiaojun Sun \ \ \ \ Priyank Kalla\\
\small Department of Electrical \& Computer Engineering\\[-0.8ex]
\small University of Utah, Salt Lake City\\
\small \texttt{\{xiaojuns,kalla\}@ece.utah.edu}
}

\date{Last Update: Feb 3, 2014}
\begin{document}
\maketitle

\section{Problem description}

In our approach, finding the complement set for varieties of ideals is necessary.
On polynomial ring $F_q[x]$, ideal $J_0$ is generated by vanishing polynomial: $J_0 = \langle f_{vanish}\rangle = \langle x^q-x\rangle$,
then we call its variety over $F_q$: $V_{F_q}(J_0)$ the \textbf{universal set} because it contains all elements in $F_q$.
Assume we have another ideal $J = \langle g\rangle$ with only 1 generator, the question is, is there any method
to find an ideal $J'$ such that
\[
V_{F_q}(J') = V_{F_q}(J_0) \setminus V_{F_q}(J) = \overline{V_{F_q}(J)}
\]

{\bf Example}:\ \ consider polynomial ring $F_4[x]$, where $\{0,1,\alpha, 1+\alpha\}$ are all elements in $F_4$.
Consider a set of 2 elements $\{1,\alpha\}$ as the variety of ideal $J$ over $F_4$:
$V_{F_4}(J) = \{1,\alpha\}$, where $J=\langle x^2+(1+\alpha)x+\alpha\rangle$. $J_0$ is the ideal generated by vanishing
polynomial $J_0 = \langle x^4-x\rangle$, then its variety over $F_4$ covers all elements in $F_4$:
\[
V_{F_4}(J_0) = V_{\overline{F_4}}(J_0) = \{0,1,\alpha,1+\alpha\}
\]
As the definition, the complement set of $V_{F_4}(J)$ is $\overline{V_{F_4}(J)} = \{0,1+\alpha\}$.
\\
{\bf Problem 1}:\ \ How to find ideal $J'$ such that $V_{F_4}(J') = \overline{V_{F_4}(J)}$?\\
{\bf Problem 2}:\ \ If $J = \langle f\rangle$ has only one generator ($f$ is univariate polynomial), does
$J'$ also has only one generator? In above example, $J = \langle x^2+(1+\alpha)x+\alpha\rangle$,
then $J' = \langle x^2+(1+\alpha)x\rangle$.
\\
{\bf Problem 3}:\ \ Please check out our conjecture in following part.

\section{A conjecture on ideal quotient}
First thing is to define \textbf{ideal quotient}.\\

({\bf Quotient of Ideals})\ \ If $I$ and $J$ are ideals in $k[x_1, \dots, x_n]$, then $I:J$
is the set
  \[
  \{f \in k[x_1, \dots, x_n] : fg \in I,\ \forall g \in J\}
  \]
and is called the {\bf ideal quotient} of $I$ by $J$.\\
\hspace{9mm}\\
Our conjecture is:\\
\textbf{Conjecture}\ \ \ \ If $J_0$, $J$ are ideals with only one univariate generator polynomial from $F_q[x]$, 
and $J_0 = \langle x^q -x \rangle$, their varieties over $F_q$ satisfy $$V_{F_q}(J_0:J) = V_{F_q}(J_0) \setminus V_{F_q}(J)$$\\
\hspace{9mm}\\
Our conjecture needs a proof. One guess may help the proof is:\\
Assume desired ideal $J' = \langle h\rangle$, vanishing polynomial is $f$ (such that $J_0 = \langle f\rangle$),
original ideal is $J = \langle g\rangle$. The variety of $J'$ must vanish $h$,
and simultaneously satisfies "vanish $J_0$ (which means it falls into $F_q$)" and "NOT vanish $J$ (means disjoint with $J$)".
\\
From these constrains we guess that
$$h = \frac{f}{g}$$
when $h=0$, it means $f=0$ and $g\neq 0$. Furthermore, apply this to example on page 1:
$$h = \frac{f}{g} = \frac{x^4-x}{x^2+(1+\alpha)x+\alpha} = x^2+(1+\alpha)x$$
and $V_{F_4}(\langle h\rangle) = \{0,1+\alpha\}$ is the complement set of $V_{F_4}(J)$.


\end{document}
