\vspace{-0.05in}
\section{Related Work}\label{sec:prev}
\vspace{-0.05in}

Contemporary high-level synthesis tools are quite adept in extracting
control/data-flow graphs (CDFGs) from the given RTL descriptions, and
also in performing scheduling, resource-sharing, retiming, and control
synthesis.  However, they are limited in their capability to employ
sophisticated algebraic manipulations to reduce the cost of the
implementation \cite{demicheli:tcad_03}. To overcome this limitation, 
there has been increasing interest in exploring the use of
algebraic manipulation for RTL synthesis of arithmetic datapaths. The
works of \cite{demicheli:iccad_98} \cite{demicheli:date_99} derive new
polynomial models of complex computational blocks for efficient
synthesis. In \cite{demicheli:tcad_03}, Symbolic Computer Algebra 
tools are used to search for a decomposition of a given polynomial
according to available library elements using a 
Gr\"{o}bner's bases based
approach. However, the derived polynomial models represent the
computations over the fields of reals ($R$), fractions ($Q$) or over
the integral domain ($Z$) - collectively called the {\it unique
factorization domains} (UFDs). This often results in a polynomial
approximation \cite{demicheli:tvlsi_01}, without properly accounting
for the effect of bit-vector size ($m$) on the resulting
computation. Moreover, Buchberger's algorithm on Gr\"{o}bner's bases
which has been used in \cite{demicheli:tcad_03} operates only on UFDs
and cannot be directly ported over non-UFDs of the type
$Z_{2^m}$. While the work of \cite{multi-word-synth} does account for
the datapath-size for allocation, it operates directly on the original
(given) arithmetic expression - thus limiting the degree of freedom in
searching for a better implementation.


%Finite rings of the type $Z_{2^m}$ are non-UFDs, due to the presence
%of nilpotent elements. 
%(An element $x$ of a ring is nilpotent if
%$x^n = 0$ for some positive integer $n$.) 
%Unfortunetly, this 
%disallows the use of fundamental computer algebra results on Euclidean
%division and factorization over non-UFDs. As a result, contemporary
%(algebra-based) high-level synthesis frameworks are limited in their
%capability to employ sophisticated algebraic manipulations to reduce
%the cost of the  implementation \cite{demicheli:tcad_03}.

Other algebraic transforms have also been explored for efficient
hardware synthesis: factorization  and common sub-expression
elimination \cite{anup:iccad_04} \cite{anup:vlsi_05}, exploiting
the structure of arithmetic circuits \cite{ienne}, term re-writing
\cite{arvind:term-rewrite}, etc. However, these techniques employ
straight-forward algebraic transforms and
also overlook the effect of bit-vector size on the given computation.

In the area of logic optimization, various spectral transforms of
boolean functions have been derived for efficient synthesis of
arithmetic circuits \cite{spectra:book}. Similar polynomial models for
random logic circuits have also been derived over Galois Fields
GF($2^m$) \cite{pradhan_galois}  \cite{pradhan_modd} \cite{galois}- so
that polynomial algebra based manipulation can be employed for logic
optimization. While these works find application at the {\it circuit-netlist
level}, they are not scalable enough to address polynomial bit-vector
computations. 


Note that our approach does not preclude some of the above mentioned
synthesis procedures \cite{anup:vlsi_05} \cite{anup:iccad_04}
\cite{multi-word-synth}; it can be combined with these approaches
as an additional optimization step. Modulo arithmetic has also been
applied to the task of circuit/RTL verification %\cite{at:zilic}
\cite{Huang:tcad01}. The concept of polynomial
functions over finite rings has also been applied to the equivalence
verification of arithmetic datapaths in \cite{iccad05}
\cite{nam:date06}.  This paper demonstrates its application to {\it
optimization} of arithmetic datapaths. 





