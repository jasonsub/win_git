\begin{abstract}
This paper addresses the problem of equivalence verification of RTL
descriptions that implement arithmetic computations (such as {\sc add,
mult}) over bit-vectors with finite widths. A bit-vector of size $m$
represents integer values from $0$ to $2^m -1$; implying that the
corresponding integer values are reduced modulo $2^m$ ($\% 2^m$). This
suggests that bit-vector arithmetic can be efficiently modeled as
algebra over finite integer rings, where the bit-vector size ($m$)
dictates the cardinality of the ring ($Z_{2^m}$). This paper model the
arithmetic datapath verification problem as equivalence testing of
polynomial functions from $Z_{2^{n_1}} \times Z_{2^{n_2}} \times
\cdots \times Z_{2^{n_d}} \rightarrow Z_{2^m}$. We formulate the
equivalence problem $f \equiv g$ into that of proving whether $f-g
\equiv 0 \% 2^m$. Fundamental concepts and results from {\it number,
ring} and {\it ideal theory} are subsequently employed to develop
systematic, complete algorithmic procedures to solve the problem. We
demonstrate application of the proposed theoretical concepts to
high-level (behavioural/RTL) verification of bit-vector arithmetic
within practical CAD settings. Using our approach, we verify a set of
arithmetic datapaths at RTL where contemporary verification approaches
prove to be infeasible.
\end{abstract}

\begin{keywords}
Bit-Vector Arithmetic, Equivalence Checking, Finite Ring Algebra,
Ideal theory
\end{keywords}

\section{Introduction}\label{sec:intro}
RTL descriptions of integer datapaths that implement polynomial
computations over bit-vectors are found in many practical designs;
{\it e.g.}, in Digital Signal Processing (DSP) for audio, video and
multimedia applications \cite{mathews:book}. The growing market for
such applications requires sophisticated CAD support for design,
analysis and verification. Such designs implement a sequence of {\sc
add, mult} type of algebraic computations over bit-vectors for
which contemporary verification frameworks do not possess the
requisite modeling and manipulation capabilities. Polynomial algebra
provides an ideal platform for modeling such arithmetic-intensive
designs. However, the word-lengths of the bit-vector variables in the
design are usually predetermined and fixed according to the desired
precision. For correct modeling of such systems, the effect of these
word-lengths need to be accounted for. For example, the largest
(unsigned) integer value that a bit-vector of size $m$ can represent
is $2^m-1$; implying that the bit-vector represents integer values
reduced modulo $2^m$ ($\% 2^m$). This suggests that bit-vector
arithmetic can be efficiently modeled as {\it algebra over finite
  integer rings of residue classes $Z_{2^m}$} \cite{allenby:book}.
This requires the use of number, ring and ideal theory concepts for
their manipulation. 

This paper addresses the problem of equivalence verification of
arithmetic datapath computations over bit-vectors where the operands
have finite bit-widths. The problem is addressed at the algorithmic
level, behavioral level or register-transfer level
(RTL). Fig. \ref{fig:matlab} depicts the overall design flow for such
applications and the corresponding verification problem as it appears
in the context of our work.

\begin{figure}[htb]
%\epsfxsize=8cm
\epsfysize=5cm
\centerline{\epsffile{figures/matlab.eps}}
\caption{The Equivalence Verification problem: Matlab to RTL design flow.}
%\vspace{-0.2in}
\label{fig:matlab}
\end{figure}

Initial algorithmic specifications (such as {\sc matlab} models) of
most signal processing applications involve data representation using
floating-point formats.  However, they are often implemented with
fixed-point architectures in order to optimize the area, delay and
power related costs of the implementations. Various automated tools
exist for this translation \cite{float-to-fixed:cases02}.
Subsequently, the fixed-point model (``specification'') can be
translated into an RTL description (``implementation''); performed
either manually, or by using automated conversion utilities, such as
\cite{mvhdl} \cite{matlab}. The resulting RTL models can also be
optimized using high-level synthesis/re-structuring operations (such
as \cite{demicheli:tcad_03} \cite{fallah-poly-factor})
\cite{sivaram:iwls06}), leading to (bit-true) equivalent RTL
descriptions. The verification problem is to prove that the
fixed-point model is computationally equivalent to the converted RTL
model or to its optimized/transformed counterpart.

Let us consider specific problem instances that necessitate the use of
finite ring algebra for equivalence testing purposes.

\subsection{Finite Word-length Bit-vector Arithmetic:} 
For polynomial datapath implementations, the design choice is often
that of a {\it uniform system word-length} for the computations
\cite{word-len-opt-hlsynth}. The datapath word-length is fixed to a
constant (say, $m$) which is defined by the desired precision. In such
cases, $m$-bit adders and multipliers produce an $m$-bit output; only
the lower $m$-bits of the outputs are used and the higher-order bits
are ignored. Usually, such computations require appropriate scaling of
coefficients and/or signals such that 'overflow' can be
avoided/ignored and standard fixed-point arithmetic can be
implemented. When the datapath size ($m$) over the entire design is
kept constant, then fixed-size bit-vector arithmetic manifests itself
as {\it polynomial algebra over the finite integer ring} of residue
classes $Z_{2^m}$; {\it i.e.}  addition and multiplication is closed
within the finite set of integers $\{0, \ldots, 2^{m} -1 \}$. In such
cases, symbolically distinct polynomials (those with different degrees
and coefficients) can become computationally (bit-true)
equivalent. The equivalence verification problem then reduces to that
of proving the {\it computational equivalence} of polynomials: $f(x_1,
x_2, \ldots, x_d) \% 2^m \equiv g(x_1, x_2, \ldots, x_d) \%2^m$, where
$f, g$ are polynomials in $d$ variables $x_1, x_2, \ldots, x_d$ and $m
=$ datapath size.

However, the fixed word-length paradigm is somewhat restrictive. Most
designs usually contain operands with different word-lengths.  For
instance, a digital audio-video mixer, may perform polynomial
arithmetic over a 20-bit audio and a 32-bit video signal
\cite{mathews:book}. As a practical example, consider the computation
performed by a digital image rejection/separation unit that takes as
input two signals: a 12-bit vector $A[11:0]$ and another 8-bit vector
$B[7:0]$. These signals are outputs of a mixer wherein one signal
emphasizes on the image signal and the other emphasizes on the desired
signal. The design produces a 16-bit output $F$. The computation
performed by the design is described in RTL as shown in
Eqn. \ref{eq:ex1}. Note that because of the specified bit-vector
sizes, the computation can be {\it equivalently} implemented as
another polynomial $G$, as shown in Eqn. \ref{eq:ex2}.\\
%
{\small
\texttt{input $A[11:0], ~B[7:0]$};\\
\texttt{output $F[15:0],~G[15:0]$};\\
\vspace{-0.1in}
\begin{eqnarray}
F &=& 16384* (A^4 + B^4) + 64767 * (A^2 - B^2) + A \nonumber \\
      & & - B + 57344 * A * B * (A - B) \label{eq:ex1} \\
G &=& 24576* A^2 *B + 15615* A^2 + 8192* A*B^2 + \nonumber \\
      & & 32768*A * B + A + 17153*B^2 + 65535*B \label{eq:ex2}
\end{eqnarray}
}
%
Such arithmetic datapaths with multiple word-length architectures can
be analyzed as polynomial functions from  $Z_{2^{n_1}}
\times Z_{2^{n_2}} \times \cdots \times Z_{2^{n_d}} \rightarrow
Z_{2^m}$ \cite{chen_96}, where $n_1, n_2, \ldots, n_d$ are the
bit-widths of input vectors and $m$ is the output width. So how do we
prove the equivalence of such computations? Efficient algorithmic
solutions to such problems is the subject of this paper.

\subsection{Problem Modeling}
We model the arithmetic computations over bit-vectors as follows. Let
$x_1, x_2, \ldots, x_d$ denote the $d$-variables (bit-vectors) in the
design. Let $n_1, n_2, \ldots, n_d$ denote the size of the
corresponding bit-vectors.  Therefore, $x_1 \in Z_{2^{n_1}}, x_2 \in
Z_{2^{n_2}}, \ldots, x_d \in Z_{2^{n_d}}$. Here, $Z_{2^{n_i}}$
corresponds to the finite set of integers $\{0, 1, \ldots, 2^{n_i} -
1\}$. Let $m$ correspond to the size of the output bit-vector $f$;
hence, $f \in Z_{2^m}$. Subsequently, we model the arithmetic datapath
computation as a {\it polynomial function} (or polyfunction) over
$Z_{2^{n_1}} \times Z_{2^{n_2}} \times \cdots \times Z_{2^{n_d}}$ to
$Z_{2^m}$ \cite{chen_96}. Here $Z_a \times Z_b$ represents the {\it
Cartesian product} of $Z_a$ and $Z_b$. In other words, the computation
is modeled as a multi-variate polynomial $F(x_1, x_2, \ldots, x_d) ~\%
2^m$, where each $x_i \in Z_{2^{n_i}}$. The equivalence problem then
corresponds to checking the congruence of two polynomials: $F ~\% ~2^m
\equiv G\%2^m$. Note that modeling fixed-size datapaths can be treated as a
special case of the above, where the bit-vectors sizes $n_1, n_2,
\ldots, n_d$ are all equal to $m$ and the polynomial is computed over
$Z_{2^m}[x_1, x_2, \ldots, x_d]$, denoted as $Z_{2^m}^d$.

\subsection{Solution Overview}
Solutions for $F(x_1, x_2, \ldots, x_d) ~\% ~2^m \equiv G (x_1, x_2,
\ldots, x_d) ~\% ~2^m$ are available for fields ($R, Q, C$), prime
rings $Z_p$, Galois fields ($GF(p^n)$), integral and Euclidean domains
- collectively called the {\it unique factorization domains (UFDs)}
\cite{allenby:book}. In our context, the finite ring formed by the
specific modulo value $2^m$ is a non-UFD, due to the presence of zero
divisors ({\it e.g.}, $4 \neq 2 \neq 0, 4\cdot 2 = 0 \%8$), and
correspondingly due to lack of multiplicative inverses. Unfortunately,
this {\it disallows} the use of many efficient factorization-based
techniques developed over UFDs.

The problem $F(x_1, \ldots, x_d)\% n = G(x_1, \ldots, x_d) \% n$ is
decidable and is shown to be NP-Hard for $n \geq 2$ \cite{ibarra}. We
transform our problem $F \%2^m \equiv G\%2^m$, into proving $(F-G)
\%2^m \equiv 0$; the well-known {\bf zero-equivalence} problem
\cite{ibarra}. For the example described in Eqns. (\ref{eq:ex1} -
\ref{eq:ex2}), we can compute $(F-G) \%2^{16}$: 
%
{\small
\begin{eqnarray}
F - G &=& 16384 (A^4 + B^4) + 32768 * A * B *(A + 1) + \nonumber\\
      & & 49152 * (A^2 + B^2)
\end{eqnarray}
}
%
Note that $(F-G)$ is a polynomial with non-zero
coefficients. However,  $ \forall A \in Z_{2^{12}}, B \in Z_{2^{8}}$,
$( F-G ) \%2^{16} = 0$; {\it i.e.}, $( F-G )$
{\it vanishes} as a function from $Z_{2^{12}} \times Z_{2^8}$ to
$Z_{2^{16}}$. Therefore, our problem reduces to that of testing
whether or not $(F-G)$ is a {\it vanishing polynomial} $\%2^m$. 

Formulating the problem in this manner ($ F-G \equiv 0$) has its
appeal because it belongs to a class of {\bf ideal membership testing}
problems. Moreover, properties of polynomial functions over finite
rings have been well studied topics in number theory and commutative
algebra \cite{niven} \cite{singmaster} \cite{chen_95} \cite{chen_96}.
This paper analyzes and extends these results (particularly those of
\cite{singmaster} and \cite{chen_96}) to derive systematic and
efficient algorithmic procedures for equivalence checking, and
demonstrates their application within a practical CAD-based
verification framework.

\subsection{Scope of the Paper} 
The approach presented in this paper has been applied to verify
high-level descriptions of arithmetic datapaths, such as those in {\sc
C} and {\sc RTL} ({\sc verilog/vhdl}), some of which were
automatically generated by {\sc Matlab} (Simulink and filter design
toolboxes) \cite{matlab}. Our technique is applicable to designs that
implement unsigned and two's complement (overflow) arithmetic. In the
DSP domain, rounding and saturation are also common modes of
approximation. Modeling such architectures as polynomial functions
over finite rings is significantly more involved and is not the
subject of this paper. For the same reason, verification of
(behavioral) RTL against its corresponding gate-level implementation
(netlist) is also not dealt with in the paper. Even within this scope,
we demonstrate that there exists a large class of applications where
our approach can very efficiently solve the problem whereas
contemporary verification techniques infeasible.

\subsection {Paper Organization} 
The paper is organized as follows: The next section reviews related
work in VLSI-CAD, symbolic and polynomial algebra. Section
\ref{sec:prelim} covers preliminary concepts regarding ring and ideal
theory and describes how our problem relates to ideal membership
testing. Section \ref{sec:univar} describes the mathematical aspects
related to proving equivalence of univariate polynomials and provides
a solution overview.  These results are extended in Section
\ref{sec:multivar} to provide an algorithm for multi-variate
computations with arbitrary input-output word-lengths. Section
\ref{sec:expt} describes the experimental setup and compares our
results with those of contemporary techniques. Finally, Section
\ref{sec:concl} concludes the paper citing future research directions.

