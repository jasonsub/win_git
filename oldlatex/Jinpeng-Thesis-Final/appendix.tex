\appendix{Canonical Forms for Polynomials over $Z_{2^{n_1}}\times Z_{2^{n_2}} \times \cdots \times
Z_{2^{n_d}} \rightarrow Z_{2^{m}}$}\label{ch:appendix}

The concept of reducibility can be also be extended to polynomials over
$Z_{2^{n_1}}\times Z_{2^{n_2}} \times \cdots \times Z_{2^{n_d}}$ to
$Z_{2^{m}}$, resulting in a minimal, canonical form. We mainly use the concepts
presented in Sec. \ref{sec:ideal_multi} to present the following result.

\begin{Theorem}\label{th:can_multi}
Let $f$ be a polyfunction from $Z_{2^{n_1}}\times Z_{2^{n_2}} \times
\cdots \times Z_{2^{n_d}}$ to $Z_{2^{m}}$. Then, $f$ can be uniquely
represented by the polynomial
\begin{equation} \label{eq:can_multi}
F(\textbf{x}) = \sum_{\textbf{k}}c_\textbf{k}{Y}_\textbf{k}(\textbf{x})
\end{equation}
where 
\begin{itemize}
\item $c_{\bf k}$ is an arbitrary integer in $Z$ such that $0 \leq c_{\bf k} < {{2^m} \over
  {(2^m,\prod_{i=1}^{d} k_i!)}}$ 
\item $\textbf{Y}_\textbf{k}$ is as defined in Def. \ref{def:multi_ff} and
\item $\textbf{k} = \langle k_1, \ldots, k_d \rangle$ for each $k_i = 0, 1, \ldots,
      \mu_i-1$ and $\mu_i$ is defined according to Eqn.(\ref{eq:mu}).
\end{itemize}
\end{Theorem}

{\bf Proof:} The proof closely follows from Theorem
\ref{th:multi}. The key concepts which are relevant to this
result are reviewed below.

Given any polynomial $F({\bf x})$ corresponding to the polyfunction
$f:Z_{2^{n_1}} \times Z_{2^{n_2}} \times \ldots Z_{2^{n_d}}
\rightarrow Z_{2^m}$, we can represent it as

\begin{equation}\label{eq:proof1_1}
F({\bf x}) = \sum _{i=1}^d Q_i Y_{\mu_i}(x_i) + \sum_{\textbf{k}}c_\textbf{k}\textbf{Y}_\textbf{k} 
\end{equation}
where, $Y_{\mu_i}(x_i)$ is the falling factorial of degree $\mu_i$ in
  variable $x_i$.

From Lemma \ref{lem:multi_rule2}, $\sum _{i=1}^d Q_i({\bf x}) Y_{{\bf
    \mu}(\textbf{i})}({\bf x}) \equiv 0 ~\% ~2^m$. This results in
    reducing the F({\bf x}) to:
\begin{eqnarray}
F({\bf x}) &=& 0 + \sum_{\textbf{k}}c_\textbf{k}\textbf{Y}_\textbf{k} \nonumber \\
           &=& \sum_{\textbf{k}}c_\textbf{k}\textbf{Y}_\textbf{k} 
\end{eqnarray}
where the degree $k_i$ of each $x_i$ is $< \mu_i$. We now analyze $F$
using Lemma \ref{lem:multi_rule3}. Let, $b_{\bf k} = {{2^m} \over
{(2^m,\prod_{i=1}^{d}k_i!)}}$, for all $k_i = 0, \ldots,
\mu_i-1$. We have three possible situations:
\begin{itemize}
\item $b_{\bf k}|c_{\bf k}$ - This indicates that $F \equiv 0 ~\% ~2^m$.
\item $b_{\bf k}$ does not divide $c_{\bf k}$ and $c_{\bf k} > b_{\bf
  k}$, then $c_\textbf{k} \textbf{Y}_\textbf{k}(\textbf{x})$ can be
  reduced by subtracting a suitable vanishing polynomial. 
\item $b_{\bf k}$ does not divide $c_{\bf k}$ and $c_{\bf k} < b_{\bf
  k}$, then $c_{\bf k}{\bf Y}_{\bf k}({\bf x})$ is already in reduced form.
\end{itemize}
{\hfill$\Box$}


\subsection{Algorithm}
We now describe an algorithm which can reduce any given polynomial
over $Z_{2^{n_1}}\times Z_{2^{n_2}} \times \cdots \times Z_{2^{n_d}}$
to $Z_{2^{m}}$ to its unique form. The algorithm takes as input $F$,
in variables $\langle x_1, x_2, \ldots, x_d \rangle$ and the
corresponding bit-widths $n_1, \ldots, n_d$.  The algorithm is given
in Fig. \ref{algo:can_redmulti} and the procedure is outlined below.

\begin{enumerate}
\item  Impose a term ordering on the  monomials of $F$ in
 degree lexicographic order.  
\item Compute $\lambda = SF(2^m)$ using the procedure from
Fig. \ref{algo:sf} and use this to obtain the $\mu_i$
values.
\item For each monomial, $mon$, in $F$,
\begin{itemize}
\item Note the value of its coefficient, $c_{\bf k}$ and the degrees
  ${\bf k} = \langle k_1,k_2,...,k_d \rangle$ of all the variables.
\item Coefficient reduction: Compute the value of $b_{\bf k}$. If $c_{\bf k} \geq
b_\textbf{k}$, then it can be reduced. Subtract the corresponding
vanishing polynomial from $mon$.
\item Degree Reduction: If $k_i \geq \mu_i$ for any $i = 1, \ldots, d$,
subtract he appropriate vanishing polynomial from $mon$.
\item Update $F$ with the reduced monomial.
\item Iteratively apply the above procedure to all monomial terms.
\end{itemize}
\end{enumerate}

\begin{figure}[ht]
  \begin{algorithm}
    \small{\bf{Multivar\_Reduce}}(\emph{F}, $d$, ${\bf x}$, $m$, ${\bf n}$)
    \  $\lambda$ = SF($2^m$)
    \ {\bf{for}} $i$ = $1$ to $d$ {\bf{do}}
    \ \ $\mu_i$ = $min\{2^{n_i}, \lambda\}$
    \ {\bf{for each}} $mon$ in \emph{F} {\bf{do}}  
    \ \ $c_{\bf k}$ = coefficient ($mon$)
    \ \ $k_i$ = Degree of $x_i$ in \emph{mon}
    \ \ $b_{\bf k} = {{2^m}\over{(2^m,\prod_{i=1}^d{k_i!}})}$ 
    \ \ {\bf{if}} ($c_{\bf k} \geq b_{\bf k}$) 
    \ \ \ $mon = mon - \lfloor{c_{\bf k}/b_{\bf k}}\rfloor \cdot a_{\bf k}\cdot {\bf Y}_{\bf k}({\bf x})$ 
    \ \ {\bf{else if}} ($k_i \geq \mu_i$ for any $i$)
    \ \ \ $mon = mon -{\bf Y}_{\bf k}({\bf x}$ 
    \ \ Update \emph{F}
    \ return \emph{F}
  \end{algorithm}
  \caption{\label{algo:can_redmulti} Algorithm Multivar\_Reduce: Reduce
  a given multi-variate polynomial $F$ to its canonical form.}
\end{figure}

\begin{Example}
Consider $F = 8x^2y$ over $Z_{2^3} \times Z_{2^4} \rightarrow
Z_{2^4}$. The algorithm reduces $F$ according to:
\begin{itemize}
\item $\lambda = SF(2^4) = 6$  and 
\begin{eqnarray}
\mu_1 &=& min\{6, 2^3\} = 6 \nonumber \\
\mu_2 &=& min\{6, 2^4\} = 6 \nonumber
\end{eqnarray}
\item Now, $mon = F = 8x^2y$.
\item The degrees $\langle k_1, k_2 \rangle$ are $\langle 2, 1
  \rangle$ and the coefficient $c_{\langle  2, 1 \rangle}$ is $8$.
\item $b_{\langle  2, 1 \rangle} = {{2^4}\over{(2^4,2!\cdot
    1!)}}$. Since $b_{\bf k} = c_{\bf k}$, the coefficient can be
    reduced according to:
\begin{eqnarray}
mon &=& 8x^2y - \lfloor{c_{\langle  2, 1 \rangle}/b_{\langle  2, 1 \rangle}}\rfloor \cdot b_{\langle  2, 1
    \rangle}\cdot Y_{\langle  2, 1 \rangle}(x, y)  ~\% ~2^4 \nonumber \\
    &=& 8x^2y - 8x(x-1)y  ~\% ~2^4 \nonumber \\
    &=& 8xy ~\% ~2^4
\end{eqnarray}
\item The canonical form of $F$ over $Z_{2^3} \times Z_{2^4}
  \rightarrow Z_{2^4}$ is $8xy$.
\end{itemize}
\end{Example}
