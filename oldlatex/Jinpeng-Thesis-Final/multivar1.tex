\section{Equivalence of Multi-variate Polynomials}\label{sec:multivar}
We now proceed to extend the results of the previous section to
polynomials in $d$ variables. In Section \ref{sec:intro}, we had shown
how the multiple-word-length bit-vector computation can be modeled as
a polynomial function from $Z_{2^{n_1}}\times Z_{2^{n_2}} \times
\cdots \times Z_{2^{n_d}}$ to $Z_{2^{m}}$.  Moreover, we had noted
that a fixed-size datapath with multiple variables is a special case
of the above, where $n_1 = \cdots = n_d = m$. Furthermore, we had
sought to exploit the concept of $SF(2^m)$ , and that of the falling
factorials $Y_k(x)$, to detect vanishing polynomials over
$Z_{2^m}[x]$. We will now extend these concepts to address
multi-variate polynomials. 

%For this purpose, we use the following
%multi-index notation: $\textbf{k} = <k_1, k_2, \ldots , k_d>$ are the
%degrees corresponding to the $d$ variables $\textbf{x} = <x_1, x_2,
%\ldots ,x_d>$, respectively. For example, the monomial ${x_1}^2{x_2}$
%can be represented in the above notation as $\textbf{x}^{\textbf{k}}
%= {x_1}^{k_1}{x_2}^{k_2}$, where $\textbf{x} = <x_1, x_2>$,
%$k_1 = 2$, $k_2 = 1$, and $\textbf{k} = <2,1>$. 

\begin{Definition}
%Falling factorials in $d$-variables can now be defined as:
Let $\textbf{k} = <k_1, k_2, \ldots , k_d> \in (Z^+)^d$, where $Z^+$
denotes the set of non-negative integers. We define,
\begin{equation}\label{eqn:falling_fact}
\textbf{Y}_\textbf{k} = \prod_{i=1}^{d}Y_{k_i}(x_i) = Y_{k_1}(x_1)
\cdot Y_{k_2}(x_2) \cdots Y_{k_d}(x_d),
\end{equation}
where $Y_{k_i}(x_i)$ is the falling factorial of degree $k_i$ in
variable $x_i$, as defined in Definition \ref{def:sk}.
\end{Definition}

Lemma \ref{lem:rule1} is now applicable over $Z_{2^m}[x_1, \ldots,
  x_d]$ as well. In other words, if a multivariate polynomial in
  $Z_{2^m}[x_1, \ldots, x_d]$ can be factorized into a product of
  $SF(2^m)$ consecutive numbers in {\it at least one of the variables}
  $x_i$, then it vanishes $\%2^m$. The following example illustrates
  this idea.

\begin{Example}\label{ex:ff1}
Consider the polynomial $F(x_1, x_2) = x_1^4 x_2 + 2x_1^3 x_2 + 3x_1^2
x_2 + 2x_1 x_2$ over $Z_{2^2}[x_1, x_2]$. Here, $SF(2^2) = 4$. 
%and the highest degrees of $x_1$ and $x_2$ are $k_1 = 4$, and $k_2 = 1$,
%respectively. 
Note that we can write %$F \%4$ can be equivalently written as 
$F = Y_{<4,1>}(x_1, x_2) \%4 = Y_{4}(x_1)\cdot Y_{1}(x_2) \%4$. Since
$F$ can be represented as a product of $4$ consecutive numbers in
$x_1$, $2^2|F$ and $F \equiv 0 \%4$. 
\end{Example}

In the above example, both the input variables $x_1, x_2$, as well as
the output $F$ are in $Z_{2^2}$. We wish to generalize these
fundamental results to analyze any arbitrary polynomial function over
$Z_{2^{n_1}} \times Z_{2^{n_2}} \times \cdots \times Z_{2^{n_d}}
\rightarrow Z_{2^{m}}$. For this purpose, another quantity, $\mu_i$,
is defined as follows \cite{chen_96}: 

%\begin{Definition}
\begin{equation} \label{eqn:mu}
\mu_i = min\{2^{n_i}, SF(2^m)\};  i = 1, 2, \ldots, d.
\end{equation}
%\end{Definition}

We now present the following results from \cite{chen_96}:
%\begin{Example}\label{ex:ff2}
%Now let $p = x^2y^2 + 3x^2y + 3xy^2 + xy$ over the same ring. Here, $k_1 =
%k_2 = 2$. Thus, $p = Y_{<2,2>}(x,y) = Y_{2}(x)\cdot Y_{2}(y)$. $p$
%cannot be factorized into a product of $SF(2^2)$ numbers in either $x$
%or $y$; it does not vanish in $Z_{2^2}$.
%\end{Example}

\begin{Lemma}\label{lem:req1}
Let $\textbf{k} = <k_1, k_2, \ldots, k_d> \in (Z^+)^d$, where $Z^+$
denotes the set of non-negative integers. Then, $\textbf{Y}_\textbf{k} 
\equiv 0$ if and only if $k_i \geq \mu_i$, for some $i$.
\end{Lemma}

\begin{Example}\label{ex:mu}
Let $F = x_1^2 x_2 - x_1 x_2$ be a polynomial corresponding to the
polyfunction  $f:Z_{2^1} \times Z_{2^2} \rightarrow Z_{2^3}$. 
%Here, $SF(2^3) = 4$, $k_1 = 2$ and $k_2 = 1$. Note that $\mu_1(2^1) = 
%min\{2^{1}, 4\} = 2 = k_1$  (the condition in Lemma \ref{lem:req1} is 
%satisfied) and $\mu_2 (2^2) = min\{2^{2}, 4\} = 4 >
%k_2$, and $F$ can be written as: 
We show that $F$ is a vanishing polynomial as $F$ can be written
according to:
\begin{eqnarray}
x_1^2 x_2 - x_1 x_2 &\equiv& x_1(x_1-1)x_2 \nonumber\\
          &\equiv& Y_{<2,1>}(x_1, x_2) \nonumber\\
          &\equiv& 0 \nonumber
\end{eqnarray}
Here, $SF(2^3) = 4$, $k_1 = 2$ and $k_2 = 1$. Note that $\mu_1 = 
min\{2^{1}, 4\} = 2 = k_1$  and $\mu_2 = min\{2^{2}, 4\} = 4 >
k_2$. Since $k_1 \geq \mu_1$, the condition in Lemma \ref{lem:req1} is
satisfied, and hence $F \equiv 0$. 
\end{Example}


\begin{Lemma}\label{lem:req2}
The expression $c_\textbf{k}\cdot \textbf{Y}_\textbf{k} \equiv 0$ if
and only if $ {{2^m} \over {
    (2^m,\prod_{i=1}^{d}k_i!)}}|c_\textbf{k}$; where:
\begin{itemize}
\item $c_\textbf{k}\in Z$;
\item $\textbf{k} = <k_1, k_2, \ldots, k_d>$ $\in (Z^+)^d$ such that 
$k_i < \mu_i, \forall i=1, \ldots, d$; and
\item ${(2^m,\prod_{i=1}^{d}k_i!)}$ is the greatest common divisor (GCD)
  of $2^m$ and $\prod_{i=1}^{d}k_i!$.
\end{itemize}

\end{Lemma}

\begin{Example}\label{ex:req2}
Consider the polynomial $F =  4x_1 x_2^2 + 4x_1 x_2$ corresponding to $f(x_1,
x_2): Z_{2^1} \times Z_{2^2} \rightarrow Z_{2^3}$. We can use Lemma
\ref{lem:req2} to prove that $f$ is a nil polyfunction.  Here,
$2^{n_1}=2$, $2^{n_2} = 4$ and $2^m = 8$. 
%\textbf{k} = $<k_1, k_2> = <1, 2>$ corresponds to the highest degrees
%of $x_1, x_2$. Moreover, $\prod_{i=1}^2 k_i! = 1!\cdot 2! = 2$. 
Also, $SF(2^m = 8) = 4$; $\mu_1 = min\{2, 4\} = 2$, $\mu_2 = min\{4,
4\} = 4$. 
%Then, $b_{<2, 1>} = {{8}\over{(8,\prod_{i=1}^2 k_i!)}} =
%{{8}\over{(8,2!\cdot 1!)}} = 4$. %Let, $a_2 = 1$. 
\vspace{-0.1in}
\begin{eqnarray}
F &\equiv& 4x_1 x_2^2 + 4x_1 x_2 \nonumber \\
  &\equiv& 4 \cdot x_1\cdot x_2\cdot(x_2-1)\nonumber\\
  &\equiv& c_{<1, 2>} \cdot Y_{<1,2>}(x_1, x_2)\nonumber\\
  &\equiv& 0 \nonumber
\end{eqnarray}
because ${{2^m} \over {(2^m,\prod_{i=1}^{d}k_i!)}} =
{{8}\over{(8,1!\cdot 2!)}} = 4 $ which divides $c_{<1, 2>} = 4$.  Also
note that here $k_1 < \mu_1$ and $k_2 < \mu_2$.
\end{Example}


Chen extended the above results to derive necessary and sufficient 
conditions for a polynomial to vanish as a function from 
$Z_{2^{n_1}} \times Z_{2^{n_2}} \times \cdots Z_{2^{n_d}}$ to
$Z_{2^m}$. We state the following theorem \cite{chen_96}: 

\begin{theorem}\label{th:red1}
Let $F$ be a polynomial representation for the function $f$ from
$Z_{2^{n_1}} \times Z_{2^{n_2}} \times \cdots \times Z_{2^{n_d}}$ to
$Z_{2^m}$. Then $F$ is a vanishing polynomial ($F \equiv 0$) if and
only if it can be represented as:
\vspace{-0.1in}
\begin{equation} \label{eq:red2}
%F = Q_\mu \textbf{Y}_\mu +
F = \sum _{i=1}^d Q_i Y_{\bf \mu(i)} +
\Sigma_{\textbf{k}}a_\textbf{k}b_\textbf{k}\textbf{Y}_\textbf{k} 
\end{equation}
where:
\begin{itemize}
\item ${\bf \mu(i)} = <0,...,\mu_i, ...,0>$ is a $d$-tuple, where
  $\mu_i$ is in position $i$ and  $\mu_i$ is defined according to
  Eqn.(\ref{eqn:mu});  
\item $Q_i \in Z[x_1, \ldots, x_d]$ are arbitrary polynomials,
  possibly zero; 
\item $Y_{\bf \mu(i)}$  is the falling factorial of degree $\mu_i$ in
  variable $x_i$; 
\item $\textbf{k} = <k_1, \ldots, k_d>$ for each $k_i = 0, 1, \ldots,
      \mu_i-1$; 
\item $\textbf{Y}_\textbf{k}$ is as defined in Eqn. (\ref{eqn:falling_fact});
\item  $a_\textbf{k} \in Z$ is an arbitrary integer and 
\item $b_\textbf{k}  = {{2^m} \over {(2^m,\prod_{i=1}^{d}k_i!)}}$. 
\end{itemize}
\end{theorem}

\begin{proof}
While a detailed proof of the above theorem is provided in
\cite{chen_96},  the theorem does follow from Lemma \ref{lem:req1} (for
each of the $d$ computations $Q_i Y_{\bf \mu(i)}$) and from Lemma
\ref{lem:req2} (for the computation
$\Sigma_{\textbf{k}}a_\textbf{k}b_\textbf{k}\textbf{Y}_\textbf{k} $).  
\end{proof}

The following example illustrates the above concept.


\begin{Example}\label{ex:red1}
Consider a polynomial $F = x_1^2 + 7x_1 + 4x_1 x_2^2 + 4x_1 x_2$ for
$f:Z_2 \times Z_{2^2} \rightarrow Z_{2^3}$. Here, $d = 2$, $\mu_1 =
min\{2, SF(8)\} = 2$; $\mu_2 = min\{2^2, SF(8)\} = 4$. Therefore,
${\bf   \mu(1)} = <\mu_1, 0> = <2, 0>$, and hence $Y_{\bf \mu(1)} =
Y_{<2,0>}(x_1,x_2)$. Similarly, ${\bf  \mu(2)} = <0, \mu_2> = <0,
4>$. Now, $F$ can be written as follows: 
\vspace{-0.1in}
\begin{eqnarray}
F  &\equiv & x_1(x_1 - 1) + 4\cdot x_1\cdot x_2\cdot (x_2 - 1)\nonumber \\
   &\equiv& Y_{<2,0>}(x_1,x_2) + a_{<1,2>}b_{<1,2>}Y_{<1,2>}(x_1,x_2)
   \nonumber \\
   &\equiv& 0 \nonumber
\end{eqnarray}
Here, $Q_1 = 1, a_{<1,2>} = 1$ and $b_{<1,2>} = 8/(8,1!\cdot 2!) =
4$. Note that $Q_2$ and all remaining $a_\textbf{k}$ terms are equal
to $0$. Hence, $F$ can be written in the form given by Theorem
\ref{th:red1}, and is thus a vanishing polynomial.
\end{Example}

Again, Eqn. \ref{eq:red2} completely describes the ideal of all
vanishing polynomials in $d$ variables over $Z_{2^{n_1}} \times
Z_{2^{n_2}} \times \cdots Z_{2^{n_d}}$ to $Z_{2^m}$. We now describe
an algorithm which determines the equivalence of any two given
polynomials $F$ and $G$ by determining if $F-G$ can be reduced to this
form, implying that $F-G \equiv 0 ~\%2^m$.

%\begin{Example}\label{ex:red2}
%Now, let $F = 3x_1 x_2^3 + 7x_1 x_2^2 + 6x_1 x_2$ be a polynomial over
%$Z_2 \times Z_{2^2} \rightarrow Z_{2^3}$. Here, $k_1 = 1$ and $k_2 =
%3$ and $F$ can be written as:
%\vspace{-0.1in}
%\begin{eqnarray}
%F & = & 3 x_1 x_2(x_2-1)(x_2-2)\nonumber\\
%  & = & 3 Y_{<1,3>}(x_1, x_2)\nonumber
%\end{eqnarray}
%Note that here the coefficient $a_\textbf{k}b_\textbf{k} = 3$, but
%$b_\textbf{k} = b_{<1,3>} = 8/(8,1!\cdot 3!)= 4$ which does not divide
%$3$. In other words, there exists no integer $a_\textbf{k}$ such that
%$a_\textbf{k} \cdot 4 = 3$. Hence the conditions of Theorem
%\ref{th:red1} is not satisfied and hence $F$ does not vanish $\%2^3$. 
%\end{Example}

\subsection{Algorithm}


The algorithm takes as input the two polynomials $F$ and $G$ in
variables $x_1, \ldots, x_d$ with corresponding bit-widths $n_1,
\ldots, n_d$. The output is {\it true} if $F \equiv G$. The algorithm
is given in Algorithm \ref{algo:multi}, the main procedure of which is
outlined below:
\begin{enumerate}
\item Find the difference of the two polynomials, $poly$. This is the
  expression which should vanish to prove equivalence. 

\item Impose a term ordering on the  monomials of $poly$, in
  descending total-degree order, with ties broken by lexicographic
  ordering of variables.  

\item Compute $SF(2^m)$. %the Smarandache function value for $2^m$.  
    Subsequently, use this to obtain the $\mu_i$ values.  
% computational procedures are outlined in \cite{smarandache}.

%\item Note the degree ($k_i$) of each variable $x_i$ in the highest order
%monomial term of the sorted polynomial.

\item Divide the polynomial by the falling factorial expressions
  $Y_{\bf \mu(i)}$ in each of the $d$ variables. %, starting from $Y_\mu$.
In any iteration, if the remainder is zero, it means that $F = \sum
  _{i=1}^d Q_i Y_{\bf \mu(i)} + 0$; hence return {\it true}. It is a
  vanishing polynomial; or $F \equiv G$. 

\item If the above process results in a non-zero remainder, we have to
  see whether the remainder can be expressed as 
$\Sigma_{\textbf{k}}a_\textbf{k}b_\textbf{k}\textbf{Y}_\textbf{k}$.
  For this purpose, use the subsequent remainder as the new $poly$. Update the
  degrees ($k_i$) and continue dividing from $Y_{\mu-1}$ to $Y_0$ for each
  variable. 
\item After each division, check for the following conditions:
\begin{itemize}
\item If the quotient can be written as $a_{\bf k}\cdot b_{\bf k}$ (where $b_{\bf k}$ is defined according to Lemma \ref{lem:req2}), and the remainder is zero, return {\it true}. It is a vanishing polynomial.
\item If the quotient can be written as $a_{\bf k}\cdot b_{\bf k}$, and the remainder is non-zero, continue to the next iteration.
\item If the quotient cannot be written as $a_{\bf k}\cdot b_{\bf k}$, return {\it false}. $poly \neq 0 \Rightarrow F \neq G$.
\end{itemize}
\end{enumerate}

%%%%%%%%%%
%{\bf Namrata - you need to fix this example to show how your monomial
%  ordering works. Note $4x_1x_2^2$ is the highest degree monomial}.

\begin{Example}\label{ex:algo}
Consider Ex. \ref{ex:red1} from the previous section. Here, $poly = x^2 +
7x + 4xy^2 + 4xy$. Let us explain the algorithmic procedure by
reducing this polynomial. Here, $2^m = 8$, $2^{n_1} = 2$ and $2^{n_2} = 4$. The
input variables are $x$ and $y$ with degrees $k_1 =2$ and $k_2 = 2$. 
\begin{itemize}
\item Initially, the monomials will be ordered as $\{ 4xy^2, x^2, 4xy, 7x\}$.
\item $SF(2^3) = 4$; $\mu_1 = min\{4, 2^1\} = 2$ and $\mu_2 =
  min\{4,2^2\} = 4$.
\item Since, $\mu_1 = k_1$, we divide $poly$ with $Y_{<2,0>}(x,y) = x(x-1)
  \% 2^3$. This gives the values, $quo = 1$ and $rem = 4xy^2 + 4xy$.
\item Now , $rem$ is non-zero, hence, $poly = rem = 4xy^2 + 4xy$. Go
  to the next step. Monomials are now ordered as $\{4xy^2, 4xy\}$.
\item Now we will consider the highest order monomial $4xy^2$, in
  which $k_1=1$ and $k_2 = 2$. Divide the new $poly$ with
  $Y_{<1,2>}(x,y) = xy(y-1) \% 2^3$. 
\item Now, $quo = 4$ and $rem = 0$. Calculate $b_{<1,2>}$ as
  ${{2^3}\over{(2^3, 1!\cdot 2!)}} = 4$. Since, $b_{<1,2>}|quo$ and $rem$ is
  zero, the algorithm returns {\it true} and ends. $poly$ is a
  vanishing polynomial.
\end{itemize}
\end{Example}


{%\ls{1}
{%\small
\begin{algorithm}
\caption{ALGORITHM MULTIVAR\_ZERO\_IDENTIFY: Determine whether the
  algebraic difference of polynomials $F$ and $G$ vanishes.}
\label{algo:multi}
\begin{algorithmic}
{\ls{1}
{\small
%\REQUIRE List of output functions and SAT requirements.
%\ENSURE Satisfiability assignments for the input variables.
\STATE {{ MULTIVAR\_ZERO\_IDENTIFY($F$, $G$, $d$, ${\bf x}$, $m$, ${\bf n}$) }}
\STATE {{ $F, G$ = Polynomials in ${\bf x}$; }}
\STATE {{ $m$ = Bit-width of $F$ and $G$; }}
\STATE {{ ${\bf x} = <x_1, x_2, \ldots, x_d>$ = Input variables; }}
\STATE {{ $<n_1, n_2, \ldots n_d>$ = Corresponding bit-widths of ${\bf x}$; }}
\STATE {{}}
%
\STATE {{$poly = F - G$; /* impose the term order */}}
\STATE {{ Compute SF($2^m$); }}
%\STATE{{ /*0=Continue; 1=Exit, it is a Zero polynomial; 2=Exit, not a Zero polynomial*/}}
%\STATE {{$isZero$ = $0$; }}\\
\STATE {{}}
%
%
\STATE{{ /* Compute values for $\mu_i$ */}}
\FOR {{ $i$ = $1$ to $d$ }}
\STATE {{ $\mu_i$ = $min\{SF(2^m), 2^{n_i}\}$; }}
\STATE {{ $k_i$ = Max. degree of $x_i$ in $poly$; }}
\ENDFOR
\STATE {{}}
%
\STATE {{ /* Check if $Y_{\bf \mu(i)}$ divides $poly$ */}}
\FOR {{ $i$ = $1$ to $d$ }}
\STATE {{ /* Lemma \ref{lem:req1} */ }}
\IF {{ ($k_i \geq \mu_i$) }}
\STATE {{ $quo, rem$ = $poly \over Y_{<0,\ldots,k_i,\ldots,0>}(x_1, \ldots, x_d)$; }}
\IF {{ ($rem == 0$) }}
\STATE {{ /* $poly = Q_i \textbf{Y}_{\bf \mu(i)}$; a vanishing polynomial */}}
\STATE {{ return $true$; }}
\ELSE
\STATE {{ $poly = rem$; }}
\STATE {{ continue; }} %break; }}
\ENDIF
\ENDIF
\ENDFOR
\STATE {{}}
%
\STATE {{ /*Iterate over all possible terms*/}}
\FOR {{ $j = \prod_{l=1}^d (\mu_l)$ to $1$ }}
%
\STATE {{ /*Update degrees*/ }}
\FOR {{ $i$ = $1$ to $d$ }}
\STATE {{ $k_i$ = Degree of $x_i$ in the highest-order monomial of $poly$; }}
\ENDFOR
%
%\STATE {{ $quo, rem = {{poly} \over {Y_{<k_1, \ldots, k_d>}(x_1, \ldots, x_d)}}$;}}
\STATE {{ $quo, rem = {{poly} \over {Y_{\bf k}({\bf x})}}$;}}
\STATE {{ $b_{\bf k}$ = $2^m\over{(2^m, \prod_{i=1}^dk_i!)}$;}}
%
\STATE {{ /*Lemma \ref{lem:req2}*/ }}
\IF {{ ($b_{\bf k}|quo$) }}
\IF {{ ($rem == 0$) }}
\STATE {{ return $true$; }}
\ENDIF
\ELSE
\STATE {{ return $false$; }}
\ENDIF
%
\ENDFOR
}
}
\end{algorithmic}
\end{algorithm}
}
}
