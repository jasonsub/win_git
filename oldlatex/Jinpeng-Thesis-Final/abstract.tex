\begin{center}{\bf ABSTRACT}\end{center}

With the spread of internet and mobile devices, transferring
information safely and securely has become more important than ever. 
Finite fields have widespread applications in such domains, such as
in cryptography, error correction codes, signal processing, 
among many others. Therefore, dedicated hardware (VLSI)
implementations of finite field arithmetic abound. In most finite
field applications, the field size  -- and therefore the bit-width of
the operands -- can be very large. For example, the U.S. National
Institute for Standards and Technology (NIST) recommends the use of
finite fields corresponding to data-path sizes of $163$-bits or more
for elliptic curve cryptography.  
The high complexity of arithmetic operations over such large fields
requires circuits to be (semi-) custom designed. This raises the
potential for errors/bugs in the implementation, which can be
maliciously exploited and can compromise the security of such
systems. Formal verification of finite field arithmetic circuits
has therefore become an imperative. 


This dissertation targets the problem of {\it formal verification of 
hardware implementations of combinational arithmetic circuits over
finite fields of the type ${\mathbb{F}}_{2^k}$}. Two specific problems
are addressed: i) verifying the correctness of a custom-designed
arithmetic circuit implementation against a given word-level
polynomial specification over ${\mathbb{F}_{2^k}}$; and ii) gate-level
equivalence checking of two different arithmetic circuit implementations. 

In practical applications, the very large size and complexity of
finite field arithmetic circuits  makes design verification a very
hard problem to solve. Contemporary automatic verification approaches,
including those that rely on solver-based technology (such as SAT and
SMT solvers), decision-diagram based methods (BDDs and BMDs), or those
based on And-Invert-Graph based reductions (AIG/ABC), etc.,
are infeasible in proving the correctness of large-scale custom designed
arithmetic circuits. In general, these techniques lack the requisite
power of abstraction and the mathematical wherewithal to efficiently
model and verify modulo-arithmetic circuits. In such cases, efficient
symbolic reasoning is required that can model and analyze the
underlying arithmetic/polynomial nature of such implementations. 

This dissertation proposes polynomial abstractions over finite fields
to model and represent the circuit constraints. Subsequently, 
decision procedures based on modern computer algebra techniques --
notably, Gr\"obner bases related theory and technology -- are
engineered to solve the verification problem efficiently. The
arithmetic circuit is modeled as a polynomial system in the ring
${\mathbb{F}}_{2^k}[x_1,x_2,\cdots,x_d]$, and computer-algebra and
algebraic-geometry based results (Hilbert's Nullstellensatz) over
finite fields are exploited for verification. Two formulations are
presented to address the implementation verification and the
equivalence checking problems.

Gr\"obner basis theory is very powerful as it allows to decide many
polynomial decision questions. However, the Gr\"obner basis
computation (Buchberger's algorithm) is known to have
double-exponential worst-case complexity in the input data. Therefore,
straight-forward use of Gr\"obner basis engines for verification is
infeasible for large circuits. To overcome this complexity, {\it we
  analyze the given circuit topology} to get more theoretical insights
into the polynomial ideals corresponding to the circuit
constraints. Based on this circuit information,  we derive efficient
{\it term orderings} to represent the polynomials. Subsequently, using
the theory of Gr\"obner bases over finite fields, we prove that our
term orderings render the set of polynomials itself a Gr\"obner basis
-- thus obviating the need for Buchberger's algorithm. We also analyze
the optimality of our term orderings and identify a minimum number of
computations required for verification via Gr\"obner basis reduction.

%Our techniques are fully automated and integrated with the symbolic
%computer algebra tool {\sc Singular}. As {\sc Singular} is a
%general-purpose computer algebra tool, its data-structures are not
%fine-tuned for circuit verification problems. To further improve our
%approach, we develop our own polynomial manipulation engines based on a
%{\it dense} representation for circuit polynomials and an $F_4$-style
%Gr\"obner Basis reduction. Tight integration of circuit analysis,
%finite field and Gr\"obner basis theory, along with an efficient
%implementation, scales verification significantly -- and these are the
%main contributions of our work.

Using our approach, experiments are performed on a variety of
custom-designed finite field arithmetic benchmark circuits. The
results are also compared against contemporary methods, based on SAT
and SMT solvers, BDDs, and AIG-based methods. Our tools can verify the
correctness of, and detect bugs in, upto $163$-bit circuits in
$\mathbb{F}_{2^{163}}$; whereas contemporary approaches are
infeasible beyond $48$-bit circuits. Experimental results are analyzed and
the advantages and limitations of our approaches are
discussed. Finally, future research directions are discussed based on
the work presented in this dissertation.

