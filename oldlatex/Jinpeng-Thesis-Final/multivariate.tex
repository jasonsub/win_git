\section{Reducibility of Multi-variate Polynomials}\label{sec:multi}
In this section, we provide the necessary theoretical foundation for our proposed algorithm to reduce the given polynomial
into a canonical representation. In short, this procedure eliminates redundancies in the RTL computations, usually
resulting in smaller, compact and efficient designs. Most of the work in this section is based on the concepts in
\cite{smarandache_function}.

\cite{smarandache_function} extended the interpretation of the Smarandache function, introduced in Section \ref{sec:sing},
to $d$ variables according to, $s_d(n) = |S_d(n)|$, where $S_d(n) = \{\textbf{k} \in N^d_0:~n ~does ~not ~divide
~\textbf{k!}\}$. For instance, in $Z_8$, $s_d(8) = |S_d(8)| = |\{0,1,2,3\}| = 4$.
We also define $\nu_2(k!)$ as the maximum degree $x$ such that $2^x$ divides $k!$. In $Z_{2^8}$, for example, $\nu_2(2^8) =
8$. Note that $\nu_2(k!)$ gives \textit{the number-of-factors-2} in $k!$.
%%%%Include a definition here%%%

Based on the above, Lemma $4$ \cite{smarandache_function} provides the necessary and sufficient conditions to reduce a given
multivariate polynomial (say $p$) in $d$ variables over $Z_{2^m}^d$. Each of its constituent monomials are successively
reduced to a unique form, such that the resulting polynomial is equivalent to the original. If the given monomial
$a$\textbf{$x^{k}$} can be reduced modulo $2^m$ then $2^m|a\textbf{k!}$. In other words, if $\nu_2$(\textbf{$2^m$}) $= m
\leq \nu_2(a) +
\nu_2(\textbf{k!})$, then the given monomial is reducible. We have the following results -
% or, the monomial is in reduced form if both $\nu_2(a) < m$ and $\nu_2(\textbf{k!}) < m$ and their sum is $< m$.

\begin{theorem}: \label{th:nu} $\nu_2(\textbf{k!}) < i$ iff $\textbf{k} \in S_d(2^i)$
\end{theorem}
\begin{proof}
$2^i|\textbf{k!}$ iff $\textbf{k!}$ has atleast $i$ $factors-of-2$. If  $\nu_2(\textbf{k!}) < i$, then the number of factors
of $2$ are lesser than $i$ or, $2^i$ does not divide $\textbf{k!}$. From the above definition of the Smarandache function,
$\textbf{k!} \in S_d(2^i)$. From Sec. \ref{sec:sing}, this means that the minimum degree of the monomial is lesser than that
required for divisibility by $2^m$. In $Z_8$, if $i=m=3$ then, for a monomial $4x^2y$, $\nu_2(2!) + \nu_2(1!) = 1 < 3$ $\Rightarrow$ $2 \in S_2(2^3)$.
\end{proof}

Let us now define the set $A_m$ as all values $\textbf{k}$, such that, $\nu_2(\textbf{k!}) < m $. In $Z_8$, $A_3 = \{0,1,2,3\}$. From Lemma $4$ stated above, $\textbf{k} \in A_m$ is one of the conditions for a reduced form. This follows directly from Theorem \ref{th:nu}. The next two results provide the conditions to check if a given monomial is in reduced form.

\begin{theorem}: If $\nu_2(\textbf{k!}) \geq m$, then the given monomial can be reduced.
\end{theorem}
\begin{proof}
This follows from Lemma 4. Also, from Theorem \ref{th:nu}, this implies that $\textbf{k} \notin S_d(2^m)$, or $2^m | a\textbf{k!}$.
\end{proof}

\begin{theorem}: If $a$\textbf{$x^{k}$} is in reduced form then $a < 2^{m-\nu_2(\textbf{k!})}$.
\end{theorem}
\begin{proof}
Assume $\nu_2(\textbf{k!}) < m$, then from Lemma $4$, $\nu_2(a) = m - \nu_2(\textbf{k!})$ for the monomial to be reducible.
The $number-of-factors-2$ less than $m$, has to be compensated for, by those in $a$. This means that, $a =
2^{m-\nu_2(\textbf{k!})}$ is reducible or $2^{m-\nu_2(\textbf{k!})}|a$. Hence $a$ has to be less than this value.
\end{proof}

What if $2^m$ does not divide $a\textbf{k!}$, but $a \geq 2^{m-\nu_2(\textbf{k!})}$? The $number-of-factors-2$ is less than
$m$ or $\nu_2(\textbf{k!}) < m$ $\Rightarrow$ $2^{m-\nu_2(\textbf{k!})}$ does not divide $a$. We write the given monomial
as, $a$\textbf{$x^k$} $= q.2^{m-\nu_2(\textbf{k!})}$\textbf{$x^k$}$ + r.$\textbf{$x^k$} where, $q.2^{m-\nu_2(\textbf{k!})}$
is reducible, while $r$ is in reduced form. We now present the algorithm for this reduction procedure implemented within MAPLE.

\subsection{Algorithm}
\begin{algo*}{PolyReduce}{poly, m}
Input: A ~multivariate ~polynomial, \textbf{poly}, ~its ~\textbf{d} ~variables ~\textbf{var_list} ~and ~its ~bit-width ~\textbf{m}
Output: Polynomial ~p ~reduced ~mod ~2^{m}.

%
%
%
%

|mon-list| \: |~Get ~the ~list ~of ~monomials ~from ~p ~in ~lexicographic ~order|

%
%
%
%Process each monomial
\FOR {~each ~monomial ~\textit{\textbf{mon}} ~in ~\textbf{\textit{mon-list}}}  ~~~~~~\@{/*Loop through all the monomials in the given polynomial*/}
~~~|flag| \: 0
~~~|a| \: |Coefficient(mon)| ~~~~~~\@{/*Get the coefficient of the monomial*/}

~~~\FOR {~each ~variable ~\textbf{\textit{v_i}} ~in ~\textbf{\textit{var-list}}}
~~~~~~K_{i} \: |Degree($v_i$) in \textbf(mon)|  ~~~~~~\@{/*Get the degree of the monomial*/}
~~~\ENDFOR
~~~|\textbf{K!}| \: \Sigma K_i!
~~~\nu (\textbf{\textit{K!}}) \: \Sigma \nu(K_i!)
~~~\alpha_{k} \: 2^{m - \nu(\textbf{K!})} - 1  ~~~~~~\@{/*Get the range of values for the coefficient of the monomial*/}

%
%
%Reduction
%
%
~~~\IF {(a > \alpha_{k})}  ~~~~~~\@{/*If the coefficient is out of range, then reduce*/}
~~~~~~|flag| \: 1
~~~~~~\IF {(a*\textbf{\textit{K!}} ~mod ~2^m = 0)}  ~~~~~~\@{/*$2^m$ divides ak!*/}
~~~~~~~~~|temp-poly| \: mon - a*\prod^{d}_{j=1}\prod^{k_d}_{i=1}(v_j+i)
~~~~~~\ELSE  ~~~~~~\@{/*$a = q*2^{m-\nu(k!)} + r$: quo and rem give the quotient and remainder*/} %when $a$ is divided by $(\alpha_{k}+1)$ */}
~~~~~~~~~|temp-poly| \: quo[{a \over (\alpha_{k}+1)}] (\alpha_k+1)(\prod^{d}_{j=1}v_j^{k_j}-\prod^{d}_{j=1}\prod^{k_j}_{i=1}(v_j+i)) + rem[{a \over (\alpha_k + 1)}]\prod^{d}_{j=1}(v_j^{k_j})
~~~~~~\ENDIF
~~~~~~|mon| \: |temp-poly|  ~~~~~~\@{/*Update the monomial*/}
~~~\ENDIF

%
%
%
%
%
~~~|reduced-poly| \: \Sigma mon ~mod ~2^m ~~~~~~\@{/*Update the polynomial*/}\\

%
%
~~~\IF{(reduced-poly)=0}
~~~~~~\RETURN ~0
~~~\ENDIF

%
%
\ENDFOR

%
%
\IF{(flag)} ~~~~~~\@{/*If any of the monomials have been reduced, recurse again to check*/}
~~~\RETURN ~|PolyReduce(reduced-poly)|
\ELSE
~~~\RETURN ~|reduced-poly| ~~~~~~\@{/*All the monomials are in reduced form-return the polynomial and stop*/}
\ENDIF
%
%
\end{algo*}
