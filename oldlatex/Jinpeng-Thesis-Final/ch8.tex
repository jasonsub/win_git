\chapter{Conclusions and Future Work}\label{ch:concl}
{\ls{1.65}{


This dissertation presents approaches to performing equivalence checking for arithmetic circuits over finite fields 
$\mathbb{F}_{2^k}$. In particular, we target two specific problems: i) verifying the correctness of a custom-designed
arithmetic circuit implementation against a given word-level polynomial specification over ${\mathbb{F}_{2^k}}$; 
and ii) gate-level equivalence checking of two structurally dissimilar arithmetic circuits. 
We propose polynomial abstractions over finite fields to model and represent the circuit constraints. 
Subsequently, decision procedures based on modern computer algebra techniques --
notably Gr\"obner bases related theory and technology -- are
engineered to solve the verification problem efficiently. 

\section{Computer Algebra Based Approaches for Equivalence Checking of Arithmetic Circuit over ${\mathbb{F}_{2^k}}$}

The arithmetic circuit is modeled as a polynomial system in the ring ${\mathbb{F}}_{2^k}[x_1,x_2,\cdots$, $x_d]$, and computer-algebra and
algebraic-geometry based results (Hilbert's Nullstellensatz) over finite fields are exploited for verification. Two formulations are
presented to address the implementation verification and the equivalence checking problems.

Using the results of Strong Nullstellensatz over finite fields, 
the first verification problem is formulated as an ideal membership testing. 
For this ideal membership test, it is required to compute a Gr\"obner basis. 
The Gr\"obner basis computation is known to have double-exponential worst-case complexity in the input data,
which makes this approach impractical. 
Therefore, straight-forward use of Gr\"obner basis engines for verification is
infeasible for large circuits. To overcome this complexity, 
we analyze the given circuit topology to get more theoretical insights
into the polynomial ideals corresponding to the circuit constraints. 
Based on this circuit information,  we derive efficient
term orderings to represent the polynomials. Subsequently, using
the theory of Gr\"obner bases over finite fields, we prove that our
term orderings render the set of polynomials itself a Gr\"obner basis
-- thus obviating the need for Buchberger's algorithm. 
To fulfill our verification purpose, we simply conduct a polynomial reduction to test whether
the equality property is a member of the ideal representing the circuit constraints.


The equivalence checking for two structurally dissimilar arithmetic circuits is still a challenge for contemporary techniques.
By utilizing computer algebra theory, we formulate this problem as a weak Nullstellensatz proof using Gr\"obner bases computation. 
Once again, this would require the computation of a reduced Gr\"obner basis, which is expensive for large circuits. 
To overcome this complexity, we want to exploit our circuit-based term ordering for polynomial representation. 
Unfortunately, unlike in the previous case, the set of polynomials corresponding to this verification instance does 
not constitute a Gr\"obner basis. Instead of computing a Gr\"obner basis for the the whole circuit, 
we identify a minimal number of S-polynomial computations that are sufficient to prove equivalence or 
to detect bugs for the whole circuit.

The verification of composite field circuits is a successful application of our computer algebra based approaches.
To construct a composite field circuit over $\mathbb{F}_{(2^m)^n}$, the finite field $\mathbb{F}_{2^k}$ is 
decomposed as $\mathbb{F}_{(2^m)^n}$, for a $k = m\cdot n$, 
and the arithmetic operations are then performed over $\mathbb{F}_{(2^m)^n}$. 
The decomposition introduces a hierarchy (modularity) in the design by lifting the ground field from $\mathbb{F}_2$
(bits) to $\mathbb{F}_{2^m}$ (words).
We formulate the verification problem as an (radical) ideal membership test at different abstraction levels.
By combining the circuit hierarchy information,  
we first verify the correctness of lower-level building-blocks (adders and 	multipliers) over the ground field $\mathbb{F}_{2^m}$; 
then verify the overall arithmetic at the higher-level over the extension field $\mathbb{F}_{(2^m)^n}$. 		

\section{Future Work}
The approaches and theories presented in this dissertation can be further extended to enhance 
the efficiency of equivalence checking of arithmetic circuits. Some future research directions are proposed here.


\subsection{Speeding up Verification using a Graphics Processing Unit}
As shown in Figure \ref{fig:2bitadder}, the equivalence of ``CIRCUIT1" and ``CIRCUIT2" 
is formulated as a single miter at word-level. However, since the circuits have multiple outputs ($k$),
we can create $k$ miters for each output bit.
In such cases, we will have to compute $Spoly(f_m,f_o) \stackrel{F,F_0}{\longrightarrow}_+ r$ 
for each of the $k$ outputs, and check if $r = 1$ in each case. 
These are going to be $n$ independent computations.
In that regard, they will immensely benefit from parallelization. 

It is desirable to implement this technique on a hardware accelerator -
particularly on a NVIDIA Graphics Processing Unit (GPU). In the
Electronic Design Automation (EDA) community, there has been a lot of
interest in exploiting GPU computing to improve synthesis and
verification algorithms. Significant speed-ups have
been observed in GPU implementation of circuit simulation algorithms
(see for example \cite{PengLi:GPU}). It is needed to further study how to
efficiently implement our circuit verification problem using
independent $S$-polynomial reductions on a general purpose GPU. 

\subsection{Extraction of Circuit Abstraction}

Suppose that we are given a circuit that implements a polynomial
function over $\mathbb{F}_{2^{k}} \rightarrow \mathbb{F}_{2^{k}}$, 
but we do not know what function does it implement. 
Can we identify a polynomial representation of this
function: $f(X, Y)$ where $X$ represents the input bit-vector and $Y$
the output? This problem is one of hierarchy abstraction and is used
in component matching and resource allocation in high-level synthesis. 

To explain this idea, let us re-visit the example of
Figure \ref{fig:2bitmul}, a 2-bit multiplier. It implements a polynomial
function $Z = A * B; ~Z, A, B  \in {\mathbb{F}}_{2^2}$. Here $A= a_0 +
a_1\alpha, B = b_0 + b_1 \alpha, Z = z_0 + z_1\alpha$. Let us
represent a polynomial for each gate in the circuit. We will impose
the following term order: {\bf lex term order} with ``circuit
Variables'' $>$ ``Inputs, A, B'' $>$ ``Output Z''. That is, we use lex
term order with $c_0 > c_1 >  c_2 >  c_3 >  r_0 > a_0 > 
a_1 > b_0 > b_1 > z_0 > z_1 > A > B > Z$. If we use this order to compute a Gr\"obner basis of the circuit
polynomials, then we obtain the following polynomials: 
\begin{eqnarray}
f_1: z_0+z_1\alpha +Z  \nonumber \\
f_2: b_0+b_1\alpha +B  \nonumber \\
f_3: a_0+a_1\alpha +A \nonumber \\
f_4: c_3+r_0+z_1  \nonumber \\
f_5: c_1+c_2+r_0  \nonumber \\
f_6: c_0+c_3+z_0  \nonumber \\
f_7:  A\cdot B+Z 		 \nonumber \\
f_8: a_1\cdot b_1+a_1\cdot B+b_1\cdot A+z_1  \nonumber \\
f_9: r_0+a_1\cdot b_1+z_1 		 \nonumber \\
f_{10}: c_2+a_1\cdot b_0  \nonumber 
\end{eqnarray}
Notice that the polynomial $f_7: A*B + Z$ is indeed the polynomial representation of the
function implemented by the circuit. And we were able to ``extract''
the polynomial representation using Gr\"obner basis. 

Polynomial interpolation techniques for this problem were studied
in  \cite{demicheli:iccad_98} \cite{demicheli:dac_99}. Further research should be conducted to  
investigate if we can use Gr\"obner basis techniques to efficiently
interpolate a polynomial representation from a circuit.


\subsection{Simulation Based Verification of Circuits}
 
In our group's previous work \cite{shekhar:tvlsi} \cite{shekhar:fmcad06}, we
show that given two polynomial functions $f, g$ over $\mathbb{Z}_{2^k}$,  exhaustive
simulation is not always necessary to prove their equivalence. We
identified an integer $\lambda$ such that functions (polynomials) $f,g$ 
need to be evaluated only for $\lambda$ inputs vectors: $\{V_1,
\dots, V_{\lambda}\}$. If $f = g$ for these $\lambda$ vectors, then $f= g$ 
over the entire design space. If $f \neq g$, then we guarantee to
catch the bug within these $\lambda$ vectors. In practice, $\lambda <<2^k$. 

Unfortunately, this result did not find much practical application as
it required that $f, g$ be polynomial functions. Not every function
(circuit) $f: \mathbb{Z}_{2^k} \rightarrow \mathbb{Z}_{2^k}$ is a polynomial
function. Instead of modeling a $k$-input/output circuit as a function
from $f: \mathbb{Z}_{2^k} \rightarrow \mathbb{Z}_{2^k}$, 
We conjecture the model can be viewed as a polynomial function over finite fields 
$f: \mathbb{F}_{2^k} \rightarrow \mathbb{F}_{2^k}$. Though this way, 
we can then prove equivalence of two polyfunctions $f, g: \mathbb{F}_{2^k} \rightarrow \mathbb{F}_{2^k}$
without resorting to exhaustive simulation. 
It is promising to solve the same problem as in \cite{shekhar:tvlsi} \cite{shekhar:fmcad06}, 
but now over a different domain: $\mathbb{F}_{2^k}$. 
