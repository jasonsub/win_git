\section{Univariate polynomials} \label{sec:sing}
In this section, we explore certain theoretical results with respect
to the properties of the ring $Z_{2^m}[x]$ and also outline a
procedure to identify univariate vanishing polynomials.

It is a well-known result in number theory that \cite{niven} for any
$k \in N$, $k!$ divides the product of $k$ consecutive numbers. For
example, $4!$ divides $4 \times 3 \times 2 \times 1$. But this is also
true of any $k$ consecutive numbers: $4!$ also divides $99 \times 100
\times 101 \times 102$. Consequently, it is possible to find the least
$n \in N$ such that $k | (n!)$. This value $n$ corresponds to the
result of the \textit{Smarandache function, Sm(k)}. For instance, $8$
divides $8 \times 7 \times 6 \times 5 \times 4 \times 3 \times 2
\times 1$. But, $8|4!$ as well, and hence, $Sm(8) = 4$. In the ring of
interest, $Z_{2^m}$, let $Sm(2^m) = k$, such that $2^m|k!$. This value
$k$ corresponds to the minimum \textit{number-of-factors-2}, required
to divide $2^m$.

To utilize this property in $Z_{2^m}[x]$, Singmaster \cite{singmaster}
proposed a set of monic polynomials, $S_k$, where each $S_i$
represents (in polynomial form) a product of $i$ consecutive numbers,
$S_0(x) = 1$, $S_1(x) = x + 1$, \ldots, $S_k(x) = (x+k)\cdot
S_{k-1}(x)$ and $s_k$ is the function determined by $S_k$. Any
expression in $Z_{2^m}[x]$ that can be factored into at least $S_{k}$
(where $k = Sm(2^m)$), will be divisible by $2^m$ and $s_k$ evaluates
to $0$ in $Z_{2^m}[x]$. Such polynomials whose functions identically
compute $0$ in the given ring are called \textit{vanishing
polynomials}.

What if the given polynomial cannot be factored into such expressions,
i.e. the \textit{number-of-factors-2} in the given function $< m$? For
example, the cubic polynomial $4x^2+4x$ in $Z_8$ can be written as
$4(x+2)(x+1)$. The missing factors, $(x+4)(x+3)$ are compensated for
by multiplying it with an appropriate constant $4$. Theorem $6$
\cite{singmaster} proposes the term $a_i \times b_i$, for a given
$S_i$ ($i<Sm(2^m)$), where $a_i, b_i \in Z_{2^m}$ and $b_i = {m\over
(i!, m)}$.

\begin{theorem} \label{uni}
 Any univariate polynomial that can be reduced to the form in Theorem
 6 \cite{singmaster} is a vanishing polynomial.
\end{theorem}
\begin{proof}
The proof follows from \cite{singmaster}.
\end{proof}

\begin{example} %%%??? ????%%%
Consider an RTL computation with one input and one output, both being $8$-bits wide. This can be expressed as a univariate polynomial, $p$, given by
\begin{center} 
$p = x^{10} + 55x^9+ 40x^8 + 230x^7 + 77x^6 + 167x^5 + 98x^4 + 156x^3 + 168x^2 + 32x$
\end{center}
Also, $Sm(2^8) = 10$. $p$ can now be factorized into a product of $10$ consecutive numbers or $S_{10}$. Thus, $p$ is a vanishing polynomial in $Z_{2^8}[x]$.
\end{example}

Niven and Warren \cite{niven} showed that it is possible to find the
ideal of vanishing polynomials in the given ring
$Z_{2^m}[x]$. Singmaster's results are based on ideal membership
testing to identify a procedure that enumerates functions evaluating
to $0$, or, in other words, check if $f(x)-g(x) \equiv 0$, for the
given functions $f(x)$ and $g(x)$. We now present results extending
the concepts in this section to RTL computations with $> 1$ inputs,
modeled as multi-variate polynomials over $Z_{2^m}^d$.

In this section, we provide the necessary theoretical foundation for our proposed algorithm to reduce the given polynomial
into a canonical representation. In short, this procedure eliminates redundancies in the RTL computations, usually
resulting in smaller, compact and efficient designs. Most of the work in this section is based on the concepts in
\cite{smarandache_function}.

\cite{smarandache_function} extended the interpretation of the Smarandache function, introduced in Section \ref{sec:sing},
to $d$ variables according to, $s_d(n) = |S_d(n)|$, where $S_d(n) = \{\textbf{k} \in N^d_0:~n ~does ~not ~divide
~\textbf{k!}\}$. For instance, in $Z_8$, $s_d(8) = |S_d(8)| = |\{0,1,2,3\}| = 4$.
We also define $\nu_2(k!)$ as the maximum degree $x$ such that $2^x$ divides $k!$. In $Z_{2^8}$, for example, $\nu_2(2^8) =
8$. Note that $\nu_2(k!)$ gives \textit{the number-of-factors-2} in $k!$.
%%%%Include a definition here%%%

Based on the above, Lemma $4$ \cite{smarandache_function} provides the necessary and sufficient conditions to reduce a given
multivariate polynomial (say $p$) in $d$ variables over $Z_{2^m}^d$. Each of its constituent monomials are successively
reduced to a unique form, such that the resulting polynomial is equivalent to the original. If the given monomial
$a$\textbf{$x^{k}$} can be reduced modulo $2^m$ then $2^m|a\textbf{k!}$. In other words, if $\nu_2$(\textbf{$2^m$}) $= m
\leq \nu_2(a) +
\nu_2(\textbf{k!})$, then the given monomial is reducible. We have the following results -
% or, the monomial is in reduced form if both $\nu_2(a) < m$ and $\nu_2(\textbf{k!}) < m$ and their sum is $< m$.

\begin{theorem}: \label{th:nu} $\nu_2(\textbf{k!}) < i$ iff $\textbf{k} \in S_d(2^i)$
\end{theorem}
\begin{proof}
$2^i|\textbf{k!}$ iff $\textbf{k!}$ has atleast $i$ $factors-of-2$. If  $\nu_2(\textbf{k!}) < i$, then the number of factors
of $2$ are lesser than $i$ or, $2^i$ does not divide $\textbf{k!}$. From the above definition of the Smarandache function,
$\textbf{k!} \in S_d(2^i)$. From Sec. \ref{sec:sing}, this means that the minimum degree of the monomial is lesser than that
required for divisibility by $2^m$. In $Z_8$, if $i=m=3$ then, for a monomial $4x^2y$, $\nu_2(2!) + \nu_2(1!) = 1 < 3$ $\Rightarrow$ $2 \in S_2(2^3)$.
\end{proof}

Let us now define the set $A_m$ as all values $\textbf{k}$, such that, $\nu_2(\textbf{k!}) < m $. In $Z_8$, $A_3 = \{0,1,2,3\}$. From Lemma $4$ stated above, $\textbf{k} \in A_m$ is one of the conditions for a reduced form. This follows directly from Theorem \ref{th:nu}. The next two results provide the conditions to check if a given monomial is in reduced form.

\begin{theorem}: If $\nu_2(\textbf{k!}) \geq m$, then the given monomial can be reduced.
\end{theorem}
\begin{proof}
This follows from Lemma 4. Also, from Theorem \ref{th:nu}, this implies that $\textbf{k} \notin S_d(2^m)$, or $2^m | a\textbf{k!}$.
\end{proof}

\begin{theorem}: If $a$\textbf{$x^{k}$} is in reduced form then $a < 2^{m-\nu_2(\textbf{k!})}$.
\end{theorem}
\begin{proof}
Assume $\nu_2(\textbf{k!}) < m$, then from Lemma $4$, $\nu_2(a) = m - \nu_2(\textbf{k!})$ for the monomial to be reducible.
The $number-of-factors-2$ less than $m$, has to be compensated for, by those in $a$. This means that, $a =
2^{m-\nu_2(\textbf{k!})}$ is reducible or $2^{m-\nu_2(\textbf{k!})}|a$. Hence $a$ has to be less than this value.
\end{proof}

What if $2^m$ does not divide $a\textbf{k!}$, but $a \geq 2^{m-\nu_2(\textbf{k!})}$? The $number-of-factors-2$ is less than
$m$ or $\nu_2(\textbf{k!}) < m$ $\Rightarrow$ $2^{m-\nu_2(\textbf{k!})}$ does not divide $a$. We write the given monomial
as, $a$\textbf{$x^k$} $= q.2^{m-\nu_2(\textbf{k!})}$\textbf{$x^k$}$ + r.$\textbf{$x^k$} where, $q.2^{m-\nu_2(\textbf{k!})}$
is reducible, while $r$ is in reduced form. We now present the algorithm for this reduction procedure implemented within MAPLE.

\subsection{Algorithm}
\begin{algo*}{PolyReduce}{poly, m}
Input: A ~multivariate ~polynomial, \textbf{poly}, ~its ~\textbf{d} ~variables ~\textbf{var_list} ~and ~its ~bit-width ~\textbf{m}
Output: Polynomial ~p ~reduced ~mod ~2^{m}.

%
%
%
%

|mon-list| \: |~Get ~the ~list ~of ~monomials ~from ~p ~in ~lexicographic ~order|

%
%
%
%Process each monomial
\FOR {~each ~monomial ~\textit{\textbf{mon}} ~in ~\textbf{\textit{mon-list}}}  ~~~~~~\@{/*Loop through all the monomials in the given polynomial*/}
~~~|flag| \: 0
~~~|a| \: |Coefficient(mon)| ~~~~~~\@{/*Get the coefficient of the monomial*/}

~~~\FOR {~each ~variable ~\textbf{\textit{v_i}} ~in ~\textbf{\textit{var-list}}}
~~~~~~K_{i} \: |Degree($v_i$) in \textbf(mon)|  ~~~~~~\@{/*Get the degree of the monomial*/}
~~~\ENDFOR
~~~|\textbf{K!}| \: \Sigma K_i!
~~~\nu (\textbf{\textit{K!}}) \: \Sigma \nu(K_i!)
~~~\alpha_{k} \: 2^{m - \nu(\textbf{K!})} - 1  ~~~~~~\@{/*Get the range of values for the coefficient of the monomial*/}

%
%
%Reduction
%
%
~~~\IF {(a > \alpha_{k})}  ~~~~~~\@{/*If the coefficient is out of range, then reduce*/}
~~~~~~|flag| \: 1
~~~~~~\IF {(a*\textbf{\textit{K!}} ~mod ~2^m = 0)}  ~~~~~~\@{/*$2^m$ divides ak!*/}
~~~~~~~~~|temp-poly| \: mon - a*\prod^{d}_{j=1}\prod^{k_d}_{i=1}(v_j+i)
~~~~~~\ELSE  ~~~~~~\@{/*$a = q*2^{m-\nu(k!)} + r$: quo and rem give the quotient and remainder*/} %when $a$ is divided by $(\alpha_{k}+1)$ */}
~~~~~~~~~|temp-poly| \: quo[{a \over (\alpha_{k}+1)}] (\alpha_k+1)(\prod^{d}_{j=1}v_j^{k_j}-\prod^{d}_{j=1}\prod^{k_j}_{i=1}(v_j+i)) + rem[{a \over (\alpha_k + 1)}]\prod^{d}_{j=1}(v_j^{k_j})
~~~~~~\ENDIF
~~~~~~|mon| \: |temp-poly|  ~~~~~~\@{/*Update the monomial*/}
~~~\ENDIF

%
%
%
%
%
~~~|reduced-poly| \: \Sigma mon ~mod ~2^m ~~~~~~\@{/*Update the polynomial*/}\\

%
%
~~~\IF{(reduced-poly)=0}
~~~~~~\RETURN ~0
~~~\ENDIF

%
%
\ENDFOR

%
%
\IF{(flag)} ~~~~~~\@{/*If any of the monomials have been reduced, recurse again to check*/}
~~~\RETURN ~|PolyReduce(reduced-poly)|
\ELSE
~~~\RETURN ~|reduced-poly| ~~~~~~\@{/*All the monomials are in reduced form-return the polynomial and stop*/}
\ENDIF
%
%
\end{algo*}
