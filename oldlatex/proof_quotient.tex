% LaTeX file for a 1 page document
\documentclass[12pt]{article}

\begin{document}

\section{Complement set for varieties of ideals with single generator}
Note: we only care about varieties (value that circuit variables can take)! 
Variety is solution to equation $poly\ generator = 0$, it is a set of elements within Galois field. 
The only reason we adopt ideal representation
is: it is convenient to represent the varieties using ideal (set of generator polynomials).\\

Back to our example. The universal set is $V(U) = \{0, 1, \alpha, 1 + \alpha\}$, 
where $U = < T^4 + T > = < f >$; and $reached$ is another set of values we can take, assume it is
$V(J) = \{1 + \alpha\}$, then $J = < T + 1 + \alpha > = < g >$. Now our objective is to find an ideal
$I$, whose variety is $V(I) = V(U) - V(J)$. Let's assume $I$ is ideal quotient of $U$ and $J$, i.e. 
$I = U:J$. Check again the definition of ideal quotient:\\

\textbf{Definition 1}\ \ \ \ If $I$, $J$ are ideals in $k[x_1,...,x_n]$, then $I:J$ is the set \\ \indent $\{f\in k[x_1,...,x_n]:fg\in I\ for \ all\ g\in J\}$.\\

Let $h^*$ be a polynomial in $I$. Then there exists one polynomial $g^* = c_1g$ from $J$ and another
polynomial $f^* = c_2f$, satisfying $c_2f = h^*\cdot c_1g$, i.e.

\[
h^* = \frac{c_2}{c_1} \frac{f}{g}
\]

So ideal $I$ only have one generator $h$ where $h = f/g$. Find varieties of $I$: make $h = 0$, which means
$f = 0\ and\ g \neq 0$. Interpret this specification: $f = 0 \to V(U), and \to \bigcap, g \neq 0 \to \overline{V(J)}$.
It means $V(U) \bigcap \overline{V(J)}$, which equals to $V(U) - V(J)$! Proof completed for single generator
ideal quotient. \\

\textbf{Theorem 1}\ \ \ \ If $I$, $J$ are ideals with only one generator, we have\\ $V(I:J) = V(I) - V(J)$.

\section{General case about using ideal quotient}
Maybe you want to expand theorem 1 to more generalized situations. Recall following theorem:\\

\textbf{Theorem 2}\ \ \ \ If I and J are ideals in $k[x_1,...,x_n]$, then $V(I \cdot J)=V(I)\bigcup V(J)$.\\

We can definitely deduce a useful conclusion from this. Say assume ideal $U = I \cdot J$ 

\end{document}
